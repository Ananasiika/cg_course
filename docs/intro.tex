\chapter*{ВВЕДЕНИЕ}
\addcontentsline{toc}{chapter}{ВВЕДЕНИЕ}

%Современная компьютерная графика широко применяется в таких областях, как кинематография и компьютерные игры. Однако, алгоритмы, используемые для создания реалистичных изображений, часто требуют больших вычислительных ресурсов. Чем более высокое качество изображения необходимо получить, тем больше времени и памяти требуется для его создания. Это становится особенно проблематичным при работе с динамическими сценами, где каждый кадр требует повторных вычислений \cite{bores}.

Цель данной работы --- разработка программного обеспечения для построения трехмерной сцены и визуализации озера с растительностью и фламинго.

Для достижения поставленной цели требуется выполнить следующие задачи:
\begin{enumerate}[label=\arabic*)]
	\item формализовать задачу в виде IDEF0 диаграммы;
	\item провести анализ существующих алгоритмов компьютерной графики: удаления невидимых линий и поверхностей, построения теней, закраски и освещения;
	\item спроектировать программное обеспечение для построения трехмерной сцены и визуализации озера с растительностью и фламинго;
	\item выбрать средства реализации спроектированного программного обеспечения и разработать его;
	\item исследовать зависимость скорости генерации кадра разработанного программного обеспечения от числа объектов на сцене и количества источников света.
\end{enumerate} 