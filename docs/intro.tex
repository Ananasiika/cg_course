\chapter*{Введение}
\addcontentsline{toc}{chapter}{Введение}

Современная компьютерная графика широко применяется в таких областях, как кинематография и компьютерные игры. Однако, алгоритмы, используемые для создания реалистичных изображений, часто требуют больших вычислительных ресурсов. Чем более высокое качество изображения необходимо получить, тем больше времени и памяти требуется для его создания. Это становится особенно проблематичным при работе с динамическими сценами, где каждый кадр требует повторных вычислений.

Цель данной работы --- реализовать построение трехмерной сцены и визуализацию озера с растительностью и фламинго.

Для достижения поставленной цели, требуется выполнить следующие задачи:
\begin{enumerate}[label=\arabic*)]
	\item описать структуру трехмерной сцены, определить объекты, из которых состоит сцена, такие как озеро, растительность и фламинго;
	\item провести анализ существующих алгоритмов построения реалистичных трехмерных изображений, исследовать преимущества и ограничения каждого алгоритма;
	\item выбрать и, если необходимо, модифицировать существующие алгоритмы трехмерной графики для создания реалистичных изображений;
	\item реализовать выбранные или модифицированные алгоритмы;
	\item создать модель трехмерного объекта фламинго;
	\item разработать программное обеспечение, которое позволит отобразить трехмерную сцену и визуализировать озеро с растительностью и фламинго, а также перемещаться по этой сцене;
	\item протестировать и оптимизировать производительность трехмерной сцены, чтобы обеспечить плавную визуализацию;
	\item провести тестирование, оценить визуальные эффекты, реалистичность и функциональность созданной трехмерной сцены озера с растительностью и фламинго.
\end{enumerate} 